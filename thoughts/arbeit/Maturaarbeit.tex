\documentclass[12pt,a4paper]{report}
\usepackage[left=2cm,right=2cm,top=2cm,bottom=2cm]{geometry}
\usepackage[onehalfspacing]{setspace}

\usepackage{ucs}
\usepackage[utf8x]{inputenc}
\usepackage[T1]{fontenc}
\usepackage[ngerman]{babel}
\usepackage{pifont}

\usepackage{graphicx}
\usepackage[export]{adjustbox}
\usepackage[hidelinks]{hyperref}
\usepackage{subcaption}
\usepackage{wrapfig}
\usepackage{mathtools}
\usepackage{listings}
\usepackage{tabto}
\usepackage{tikz}
\usetikzlibrary{matrix,chains,positioning,decorations.pathreplacing,arrows,calc}

\begin{document}

\begin{titlepage}
	\centering
	{\Large Gymnasium Bäumlihof, 5Bb \par}
	\vspace{1cm}
	{\LARGE\scshape Maturaarbeit\par}
	\vspace{1.5cm}
	{\huge\bfseries Kann der Computer Werbung erkennen?\par}
	\vspace{0.6cm}
    {\Large Bilderkennung mit einem Neuronalen Netzwerk\par}
	\vspace{2cm}
	{\Large\itshape Georg Schwan\par}
	\vfill
	Betreuungsperson\par
	{\itshape Test1\par}
	Korreferent\par
	{\itshape Test2}
	\vfill
	{\large Datum\par}
\end{titlepage}

\tableofcontents

\newpage

\chapter{Einleitung}\label{ch:einleitung}

\section{Motivation}
\label{sec:motivation}

\section{Aufbau der Arbeit}
\label{sec:aufbauDerArbeit}

\chapter{Neuronales Netzwerk}
\label{ch:neuronalesNetzwerk}

\section{Konzept}\label{sec:konzept}
Ein Neuronales Netzwerk wird nicht gezeigt wie man ein Problem löst, sondern man zeigt es viele Beispiele und lässt es
seine eigene Lösung finden.
Alles was ein Neuronales Netzwerk braucht sind Beispiel Daten, bei denen wir schon das korrekte Ergebnis wissen,
so dass man sehen kann ob die Prognose des Neuronales Netzwerks korrekt oder inkorrekt ist und man es daraufhin anpassen kann.
Zum Beispiel, wir wollen einem Netzwerk beibringen ob in einem Bild ein Auto vorkommt,
dazu speisen wir dem Neuronale Netzwerk viele Bilder, mit und ohne Auto, ein
Mit jedem Bild, dass das Neuronale Netzwerk falsch hat, können wir nun das Netzwerk anpassen
bis es immer besser wird im erkennen eines Autos.

\section{Architektur}\label{sec:architektur}
User Gehirn kann Entscheidungen treffe, da wir billionen von Neuronen haben, die miteinander verbunden sind und sich
verständigen können.
Aber ein Neuron an sich ist praktisch nutzlos, aber in grosser Anzahl können sie komplexeste Probleme lösen.

Nach dem gleichen Prinzip funktioniert ein Neuronales Netzwerk
Es besteht aus vielen Neuronen (daher der Name) die miteinander verbunden sind.


\begin{figure}[!h]
    \centering
\begin{tikzpicture}[
init/.style={
  draw,
  circle,
  inner sep=2pt,
  font=\Huge,
  join = by -latex
},
squa/.style={
  draw,
  inner sep=2pt,
  font=\Large,
  join = by -latex
},
start chain=2,node distance=13mm
]
\node[on chain=2]
  (x2) {$x_2$};
\node[on chain=2,join=by o-latex]
  {$w_2$};
\node[on chain=2,init] (sigma)
  {$\displaystyle\Sigma$};
%\node[on chain=2,squa,label=above:{\parbox{2cm}{\centering Activate \\ function}}]
%  {$f$};
\node[on chain=2,label=above:Ausgabe,join=by -latex]
  {$y$};
\begin{scope}[start chain=1]
\node[on chain=1] at (0,1.5cm)
  (x1) {$x_1$};
\node[on chain=1,join=by o-latex]
  (w1) {$w_1$};
\end{scope}
\begin{scope}[start chain=3]
\node[on chain=3] at (0,-1.5cm)
  (x3) {$x_3$};
\node[on chain=3,label=below:Gewichte,join=by o-latex]
  (w3) {$w_3$};
\end{scope}
\node[label=above:\parbox{2cm}{\centering Verzerrung \\ $b$}] at (sigma|-w1) (b) {};

\draw[-latex] (w1) -- (sigma);
\draw[-latex] (w3) -- (sigma);
\draw[o-latex] (b) -- (sigma);

\draw[decorate,decoration={brace,mirror}] (x1.north west) -- node[left=10pt] {Eingabe} (x3.south west);
\end{tikzpicture}
    \caption{Einzelner Neuron in einem Neuronalen Netzwerks}
    \label{fig:neuron1}
\end{figure}

Ein Neuron in einem Neuronales Netzwerk wird als Mathematische funktion definiert wie in Abbildung~\ref{fig:neuron1}
zu erkennen ist.
Ein Neuron hat $n$ verschiedene Eingaben, die als $x_j$ bezeichnet werden und mit einem spezifischen Gewicht $w_j$ multipliziert werden.
Die Ausgabe erfolgt indem man alle gewichteten Eingaben, mit einer Verzerrung $b$, addiert.
Als Ausdruck:
\[y =\sum_{j=1}^{n} x_j * w_j + b\]
Die Gewichte $w_j$ und die Verzerrung $b$ des Neurons sind die Parameter, die angepasst werden und somit das Neuron lernfähig machen.

Wie auch im biologischen Gehirn ist ein Neuron allein nutzlos.
Erst wenn man die Neuronen miteinander verbindet kann es komplexe Zusammenhänge modellieren.
\begin{figure}[h]
    \centering
\begin{tikzpicture}[
plain/.style={
  draw=none,
  fill=none,
  },
net/.style={
  matrix of nodes,
  nodes={
    draw,
    circle,
    inner sep=10pt
    },
  nodes in empty cells,
  column sep=2cm,
  row sep=-9pt
  },
>=latex
]
\matrix[net] (mat)
{
|[plain]| \parbox{1.3cm}{\centering} & |[plain]| \parbox{1.3cm}{\centering} & |[plain]| \parbox{1.3cm}{\centering} \\
& |[plain]| \\
|[plain]| & \\
& |[plain]| \\
  |[plain]| & |[plain]| \\
& & \\
  |[plain]| & |[plain]| \\
& |[plain]| \\
  |[plain]| & \\
& |[plain]| \\    };
\foreach \ai [count=\mi ]in {2,4,...,10}
  \draw[<-] (mat-\ai-1) -- node[above] {Eingabe \mi} +(-2.8cm,0);
\foreach \ai in {2,4,...,10}
{\foreach \aii in {3,6,9}
  \draw[->] (mat-\ai-1) -- (mat-\aii-2);
}
\foreach \ai in {3,6,9}
  \draw[->] (mat-\ai-2) -- (mat-6-3);
\draw[->] (mat-6-3) -- node[above] {Ausgabe} +(2.5cm,0);
\end{tikzpicture}
    \caption{Mögliche architektur eines Neuronalen Netzwerk}
    \label{fig:network1}
\end{figure}

Eine mögliche Architektur kann wie in  Abbildung~\ref{fig:network1} ausschauen.
Ein Netzwerk wird generell immer in verschiedene Schichten unterteilt.
Die linke Schicht wird als eingabe Schicht bezeichnet und die Neuronen in dieser Schicht werden eingabe Neuronen genannt.
Analog da zu wird die rechte Schicht ausgabe Schicht genannt, die die ausgabe Neuronen beinhaltet.
Die mittleren Schichten, die von der Anzahl variieren können, werden versteckte Schichten genannt.
Die Anzahl der Neuronen in jeder Schicht kann auch variieren.
Abbildung~\ref{fig:network2} zeigt eine andere mögliche Architektur für ein Netzwerk, welches 2 versteckte
Schichten hat.
\begin{figure}[h]
    \centering
\begin{tikzpicture}[
plain/.style={
  draw=none,
  fill=none,
  },
net/.style={
  matrix of nodes,
  nodes={
    draw,
    circle,
    inner sep=10pt
    },
  nodes in empty cells,
  column sep=2cm,
  row sep=-9pt
  },
>=latex
]
\matrix[net] (mat)
{
|[plain]|  & |[plain]|  & |[plain]|  & |[plain]| \\
& |[plain]|& |[plain]|& |[plain]| \\
|[plain]| && |[plain]|& |[plain]| \\
& |[plain]|&& |[plain]|& |[plain]| \\
  |[plain]| && |[plain]|&& |[plain]| \\
& |[plain]| & \\
  |[plain]| && |[plain]|&& |[plain]| \\
& |[plain]| && |[plain]|& |[plain]|  \\
  |[plain]| & & |[plain]|& |[plain]|  \\
& |[plain]|& |[plain]|& |[plain]| \\     };
\foreach \ai in {2,4,...,10}
{\foreach \aii in {3,5, 7,9}
  \draw[->] (mat-\ai-1) -- (mat-\aii-2);
}
\foreach \ai in {3,5,7,9}
{\foreach \aii in {4,6,8}
  \draw[->] (mat-\ai-2) -- (mat-\aii-3);
}
\foreach \ai in {4,6,8}
{\foreach \aii in {5, 7}
  \draw[->] (mat-\ai-3) -- (mat-\aii-4);
}
\draw[decorate,decoration={brace}] (mat-1-1.west) -- node[above=5pt] {\parbox{3cm}{\centering Eingabe\\Schicht}} (mat-1-1.east);
\draw[decorate,decoration={brace}] (mat-1-2.west) -- node[above=5pt] {\parbox{3cm}{\centering Versteckte\\Schichten}} (mat-1-3.east);
\draw[decorate,decoration={brace}] (mat-1-4.west) -- node[above=5pt] {\parbox{3cm}{\centering Ausgabe\\Schicht}} (mat-1-4.east);
\draw[decorate,decoration={brace,mirror}] ($(mat-1-1.west)+(-0.1cm,-0.2cm)$) -- node[left=5pt] {\parbox{2cm}{\centering Eingabe\\Neuronen}} ($(mat-10-1.west) + (-0.1cm,-16pt)$);
\draw[decorate,decoration={brace}] ($(mat-5-4.east)+(+0.1cm,+16pt)$) -- node[right=5pt] {\parbox{2cm}{\centering Ausgabe\\Neuronen}} ($(mat-7-4.east) + (+0.1cm,-16pt)$);
\end{tikzpicture}
    \caption{Neuronales Netzwerk mit 2 versteckten Schichten}
    \label{fig:network2}
\end{figure}
Jedes Neuron von der vorigen Schicht ist mit jedem Neuron der nachfolgenden Schicht verbunden.
Dies ist ein klassisches vorwärtsgekoppeltes Netzwerk (im englischen feed forward network).
Wichtig zu beachten ist, dass keine Schleifen vorkommen.
Es gibt Netzwerke wo Schleifen vorkommen, aber auf diese wir nicht näher eingegangen:

\chapter{Lösungsansatz}
\label{ch:lösungsansatz}

\chapter{Umsetzung}
\label{ch:umsetzung}

\chapter{Reflexion}
\label{ch:reflexion}

%\clearpage
%\phantomsection
%\addcontentsline{toc}{chapter}{Literaturverzeichnis}
%\nocite{*}
%\bibliographystyle{plain}
%\bibliography{literatur}
%
%\clearpage
%\phantomsection
\addcontentsline{toc}{chapter}{Abbildungsverzeichnis}

\listoffigures

\appendix

\chapter*{Ehrlichkeitserklärung}

Die eingereichte Arbeit ist das Resultat meiner persönlichen, selbstständigen Beschäftigung mit dem Thema.
Ich habe für sie keine anderen Quellen benutzt als die in den Verzeichnissen aufgeführten.
Sämtliche wörtlich übernommenen Texte (Sätze) sind als Zitate gekennzeichnet.

\vspace{2cm}
Insert Datum
\end{document}
